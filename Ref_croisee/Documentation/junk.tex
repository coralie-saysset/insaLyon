\documentclass{article}

\usepackage[utf8]{inputenc}
\usepackage[francais]{babel}
\usepackage{amsmath}
\usepackage{amssymb}
\usepackage{fancyhdr}
\pagestyle{fancy}
\usepackage{graphicx}

\lhead{Coralie Saysset - Romain Gerard}
\chead{}
\rhead{B3301}


\begin{document}

\newpage
  
  \begin{subsection}{Test \no 1}
    \begin{paragraph}{Descriptif :}
      Le test \no 1 réalise le premier exemple donné par le sujet.\\ Les fichiers analysés sont : 
      
      \begin{listing}[h!]
      \centering{file1.cpp}
      \begin{minted}[linenos,
		     gobble=2]{cpp}
	  // affiche le message "Hello world"
	  int main() {
	      cout <<"Hello world"<<endl;
	      cout<<endl;
	      return 0;
	  }
      \end{minted}
      \end{listing}
      
      \begin{listing}[h!]
      \centering{file1.h}
	\begin{minted}[linenos,gobble=2]{cpp}
	   int main();
	\end{minted}
      \end{listing}
      
        \begin{listing}[h!]
        \centering{key1.txt}
	\begin{minted}[linenos,gobble=2]{cpp}
	   int
	   world
	   template
	\end{minted}
      \end{listing}
      
    \end{paragraph}
    
    \begin{paragraph}{Résultat attendu :} 
      Nous lançons le programme avec le contexte suivant :  
       \begin{center}
	\textbf{$tp\_stl\ -e\ -k\ key1.txt\ file1.cpp\ file1.h$}
      \end{center}
      
      Nous devons obtenir le résultat ci-dessous après avoir trié la sortie : 
      \begin{listing}[h!]
          \centering{file1.res}
      \begin{minted}[linenos,
		     gobble=2]{pascal}
	  cout	file1.cpp 3 4	
	  endl	file1.cpp 3 4	
	  main	file1.cpp 2	file1.h 1	
	  return	file1.cpp 5	
      \end{minted}
  
      \end{listing}
    \end{paragraph}

 \end{subsection}
     
   \newpage 
       
  \begin{subsection}{Test \no 2}
  
    \begin{paragraph}{Descriptif :}
	Le test \no 2 réalise le deuxième exemple donné par le sujet.\\ Les fichiers analysés sont : 
      
      \begin{listing}[h!]
      \centering{file2.cpp}
      \begin{minted}[linenos,
		     gobble=2]{cpp}
	  // affiche le message "Hello world"
	  int main() {
	      cout <<"Hello world"<<endl;
	      cout<<endl;
	      return 0;
	  }
      \end{minted}
      \end{listing}
      
      \begin{listing}[h!]
      \centering{file2.h}
	\begin{minted}[linenos,gobble=2]{cpp}
	   int main();
	\end{minted}
      \end{listing}
      
        \begin{listing}[h!]
        \centering{key2.txt}
	\begin{minted}[linenos,gobble=2]{cpp}
	   int
	   world
	   template
	\end{minted}
      \end{listing}
      
    \end{paragraph}
	   
    \begin{paragraph}{Résultat attendu :}
       Nous lançons le programme avec le contexte suivant après avoir trié la sortie :  
       \begin{center}
	\textbf{$tp\_stl\ -k\ key2.txt\ file2.cpp\ file2.h$}
      \end{center}
      
      Nous devons obtenir le résultat ci-dessous : 
      \begin{listing}[h!]
          \centering{file2.res}
      \begin{minted}[linenos,
		     gobble=2]{pascal}
	  int	file2.cpp 2	file2.h 1	
      \end{minted}
  
      \end{listing}
    \end{paragraph}

  \end{subsection}

   \newpage 
  
  \begin{subsection}{Test \no 3}
    
    \begin{paragraph}{Descriptif :}
	Le test \no 3 réalise le test sur le fichier main de notre programme.\\ Les fichiers analysés sont : 
      
      \begin{minted}[linenos,
		     gobble=2]{cpp}
	    //============================================================================
	    // Name        : Ref_croisee.cpp
	    // Author      :
	    // Version     :
	    // Copyright   : Your copyright notice
	    // Description : Hello World in C++, Ansi-style
	    //============================================================================

	    #include <iostream>
	    #include <vector>

	    #include "CmdLine/cmdLine.hpp"
	    #include "References/Referenceur.hpp"
	    #include "References/References.hpp"

	    using namespace std;
	    using namespace Reference_croisee;

	    int main( int argc, char** argv )
	    {/*{{{*/

		CmdLine::Arguments args;
		CmdLine::Parser parser( "Permet de referencer des mots clefs a travers des fichiers" );
		parser.addOption( "exclude,e",  "Inverse le fonctionnement du programme" );
		parser.addOption( "keyword,k",  "Specifie la liste des mots clefs a utiliser", true );

		try {
		    parser.parse( argc, argv, args );

		} catch( exception& e ) {
		    cout << "Une erreur c'est produit durant la recuperation de la ligne de commande : "
			 << endl << e.what() << endl;
		}

		//----------------------------------------------------------------------
		//  On charge les fichiers a referencer
		//----------------------------------------------------------------------
		vector<string> ficsReferencer;

		if( args.count( "__args__" ) ) {
		    ficsReferencer = args.get<vector<string> >( "__args__" );

		} else {
		    cerr << "Aucun fichier a referencer !" << endl;
		    return 1;
		}

		//----------------------------------------------------------------------
		//  On charge les mots clefs si ils sont fournis
		//----------------------------------------------------------------------
		string fichierMotClef;

		if( args.count( "keyword" ) ) {
		    fichierMotClef = args.get<string>( "keyword" );
		}

		//----------------------------------------------------------------------
		//  L'etat dans lequel mettre le programme
		//----------------------------------------------------------------------
		bool mode( args.count( "exclude" ) );


		References refs;

		//----------------------------------------------------------------------
		//  On effectue la reference croisee
		//----------------------------------------------------------------------
		try {
		    Referenceur referenceur( fichierMotClef, mode );
		    referenceur.referencer( ficsReferencer, refs );

		} catch( exception& e ) {
		    cerr << "Une erreur est survenue durant la reference croisee : " << endl;
		    cerr << e.what() << endl;
		}


		//----------------------------------------------------------------------
		//  On affiche les resultats
		//----------------------------------------------------------------------
		refs.display( cout );

		return 0;
	    }/*}}}*/
      \end{minted}
      
    \end{paragraph}
    
    \newpage 
    
    \begin{paragraph}{Résultat attendu :}
       Nous lançons le programme avec le contexte suivant :  
       \begin{center}
	\textbf{$tp\_stl\ file3.cpp$}
      \end{center}
      
      Nous devons obtenir le résultat ci-dessous après avoir trié la sortie: 
      \begin{listing}[h!]
          \centering{file3.res}
      \begin{minted}[linenos,
		     gobble=2]{pascal}
	  bool	file3.cpp 60	
	  catch	file3.cpp 30 72	
	  char	file3.cpp 19	
	  cout	file3.cpp 31 81	
	  else	file3.cpp 43	
	  if	file3.cpp 40 53	
	  int	file3.cpp 19 19	
	  namespace	file3.cpp 16 17	
	  return	file3.cpp 45 83	
	  true	file3.cpp 25	
	  try	file3.cpp 27 68	
	  using	file3.cpp 16 17		
      \end{minted}
  
      \end{listing}
    \end{paragraph}
    
    
  \end{subsection}

  \newpage

  \begin{subsection}{Test \no 4}
    \begin{paragraph}{Descriptif :}
      Le fichier à analyser est le même que dans le test précédent, seul le contexte d'execution change.
    \end{paragraph}
    
\begin{paragraph}{Résultat attendu :}
       Nous lançons le programme avec le contexte suivant :  
       \begin{center}
	\textbf{$tp\_stl\ file4.cpp$}
      \end{center}
      
      Nous devons obtenir le résultat ci-dessous après avoir trié la sortie: 
      \begin{listing}[h!]
          \centering{file4.res}
      \begin{minted}[linenos,
		     gobble=2]{cpp}
	  addOption	file4.cpp 24 25	
	  argc	file4.cpp 19 28	
	  args	file4.cpp 22 28 40 41 53 54 60	
	  Arguments	file4.cpp 22	
	  argv	file4.cpp 19 28	
	  cerr	file4.cpp 44 73 74	
	  CmdLine	file4.cpp 22 23	
	  count	file4.cpp 40 53 60	
	  display	file4.cpp 81	
	  e	file4.cpp 30 32 72 74	
	  endl	file4.cpp 31 32 44 73 74	
	  exception	file4.cpp 30 72	
	  fichierMotClef	file4.cpp 51 54 69	
	  ficsReferencer	file4.cpp 38 41 70	
	  get	file4.cpp 41 54	
	  main	file4.cpp 19	
	  mode	file4.cpp 60 69	
	  parse	file4.cpp 28	
	  Parser	file4.cpp 23	
	  parser	file4.cpp 23 24 25 28	
	  Reference_croisee	file4.cpp 17	
	  referencer	file4.cpp 70	
	  References	file4.cpp 63	
	  Referenceur	file4.cpp 69	
	  referenceur	file4.cpp 69 70	
	  refs	file4.cpp 63 70 81	
	  std	file4.cpp 16	
	  string	file4.cpp 38 41 51 54	
	  vector	file4.cpp 38 41	
	  what	file4.cpp 32 74	
      \end{minted}
  
      \end{listing}
    \end{paragraph}
  \end{subsection}





\end{document}