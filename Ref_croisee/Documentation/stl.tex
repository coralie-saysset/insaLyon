\documentclass{article}

\usepackage[utf8]{inputenc}
\usepackage[francais]{babel}
\usepackage{amsmath}
\usepackage{amssymb}
\usepackage{fancyhdr}
\pagestyle{fancy}
\usepackage{graphicx}
\usepackage{minted}

\lhead{Coralie Saysset - Romain Gerard}
\chead{}
\rhead{B3301}


\begin{document}

\begin{center} 
\Huge{Références croisées \\ Spécification et Conception}
\end{center}



%-----------------------------------------------------------------------------------------------------------------------------------%
%	Spécification compléte
%-----------------------------------------------------------------------------------------------------------------------------------%
\begin{section}{Spécification complète}


  \begin{subsection}{Définitions}

    \begin{itemize}
      \item \textbf{ligne} : Une suite de caractères terminées par un retour chariot

      \item \textbf{Identificateur} : Mot sensible à la casse composé uniquement de caractères alphanumériques et du caractère `\_`. 
				      Les commentaires ou chaines litérales ne peuvent contenir d'identifcateurs

      \item \textbf{Délimiteur} : Un caractère représentant une séparation entre deux mots (Ex: une virgule, un espace, un point, ...) 

      \item \textbf{Référence croisée} : Fait de rechercher un identificateur dans un ou plusieurs fichiers sources pour déterminer sa localisation
    \end{itemize}

  \end{subsection}


  \begin{subsection}{Description du programme}
    Le but du programme est de permettre de retrouver rapidement l'emplacement d'identifcateurs dans une collection de fichiers.
    On cherche à connaitre dans quel(s) fichier(s) et à quelle(s) lignes les identifcateurs apparaissent.
    
    Dans le cas où un identifacteur apparaitrait plusieurs fois sur une même ligne, nous avons pris la décision d'afficher
    la ligne conserné autant de fois qu'il y a d'occurences. 
    
    Ex: Pour le code ci-dessous présent dans le fichier ``test.cpp'' et avec comme identifacteur la lettre i 
    
    \begin{minted}{cpp}
      for( int i = 0; i < 42; i++ );
    \end{minted}
    Le programme produira la sortie suivante : 

    i$\xrightarrow{}$test.cpp$\bullet$1$\bullet$1$\bullet$1$\downarrow$ \\
    
    Par defaut les identificateurs sont les mots clefs utilisés par le langage C++. Il est cependant possible de spécifier
    un fichier en argument du programme pour définir précisement quelles seront les identifcateurs recherchés par la référence
    croisée. Le fichier d'identifacteurs ne devra contenir qu'un seul identifacteur valide par ligne. Le programme présuposera que
    le fichier fournit en argument respecte ce formalisme.
    
    Le programme disposera également de la fonctionnalité permettant d'exclure une liste d'identifacteurs.
    
  \end{subsection}

  
  \newpage
  \begin{subsection}{Spécifications des options}
    \begin{center}
      \textbf{$tp\_stl\ [-e]\ [-k fichier\_mot\_clef]\ [nomfichier]+$}
    \end{center}

    \begin{itemize}
      \item[] \textbf{-e} : Permet d'inverser le comportement par défaut du programme. Exclut de la référence croisée tous les mots clefs
      
      \item[] \textbf{-k fichier\_mot\_clef} : Permet de spécifier au programme une liste d'identifacateurs à rechercher par la référence croisée

      \item[] \textbf{nomfichier} : Chemin vers un ou plusieurs fichiers où rechercher les identifacateurs
    \end{itemize}

  \end{subsection}

\end{section}



%-----------------------------------------------------------------------------------------------------------------------------------%
%	Tests fonctionnel
%-----------------------------------------------------------------------------------------------------------------------------------%
\begin{section}{Tests fonctionnel}


  \begin{subsection}{Test \no 1}
  \end{subsection}


  \begin{subsection}{Test \no 2}
  \end{subsection}

\end{section}





%-----------------------------------------------------------------------------------------------------------------------------------%
%	Architecture générale
%-----------------------------------------------------------------------------------------------------------------------------------%
\begin{section}{Architecture générale}

  \begin{subsection}{Diagramme de classe}
  \end{subsection}

  \begin{subsection}{Commentaires}
  \end{subsection}

\end{section}


%-----------------------------------------------------------------------------------------------------------------------------------%
%	Algorithmes principaux
%-----------------------------------------------------------------------------------------------------------------------------------%
\begin{section}{Algorithmes principaux}

  \begin{subsection}{Parseur de ligne de commandes}
  \end{subsection}

  \begin{subsection}{Parseur des fichiers C++}
  \end{subsection}

\end{section}



%-----------------------------------------------------------------------------------------------------------------------------------%
%	Structures de données
%-----------------------------------------------------------------------------------------------------------------------------------%
\begin{section}{Analyse critique des structures de données}

  \begin{subsection}{Structure des identificateurs}
  \end{subsection}

  \begin{subsection}{Structure des occurences}
  \end{subsection}

\end{section}

\end{document}