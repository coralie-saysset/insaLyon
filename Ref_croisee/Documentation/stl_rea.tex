\documentclass{article}

\usepackage[utf8]{inputenc}
\usepackage[francais]{babel}
\usepackage{amsmath}
\usepackage{amssymb}
\usepackage{fancyhdr}
\pagestyle{fancy}
\usepackage{graphicx}
\usepackage{placeins}
\usepackage{minted}




\lhead{Coralie Saysset - Romain Gérard}
\chead{}
\rhead{B3301}


\begin{document}

\fancyfoot[]{}
\fancyfoot[R]{\ \\Insa de Lyon\\Dpt. Informatique}
\fancyfoot[C]{\ \\\ \\ 2011}
\fancyfoot[L]{Tuteurs :\\\ Nabila Benharkat\\\ Eric Guerin}
\begin{center}
\Huge{\ \\[6em]Références croisées \\ Document de réalisation}
\end{center}

\newpage

\fancyfoot[]{}
\setcounter{tocdepth}{4}
\tableofcontents

\newpage

\fancyfoot[C]{\thepage}


\begin{section}{Structure de données de la STL}

  \begin{subsection}{Choix pour les identificateurs}
  Les tables de hashages n'existant pas dans le standard C++03 nous avons décidé d'utiliser la classe \textbf{unordered\_map} présente 
  dans le standard C++ (C++1x). Cette classe est présente dans le namespace tr1 (ToRealease) et implémente les tables de hashages
  pour les classes standards de la STL (comme string).

  \begin{center}
	\textbf{$tr1::std::unordered\_map<std::string, \{Occurences\}>$}
      \end{center}
  \end{subsection}

  \begin{subsection}{Choix pour les références d'occurences}
  Nous avons décidé d'utiliser une \textbf{map} de \textbf{list} pour stocker les occurences des mots clefs.
  La clef de la structure map est le nom du fichier où apparait l'occurence et la valeur est une liste d'entier indiquant
  le numéro des lignes où l'on trouve le mot clef.

  \begin{center}
	\textbf{$std::map<std::string, list<int>>$}
      \end{center}
  \end{subsection}

\end{section}


\newpage 
\oddsidemargin = 15pt
\begin{section}{Réalisation des tests}

Pour lancer l'execution des tests, il faut se rendre dans le répertoire des sources et taper dans un terminal la commande ``make test''
 \begin{subsection}{Test \no 1}
    \begin{paragraph}{Descriptif :}
      Le test \no 1 réalise le premier exemple donné par le sujet.\\ Les fichiers analysés sont : 
      
      \begin{listing}[h!]
      \centering{file1.cpp}
      \begin{minted}[linenos,
		     gobble=2]{cpp}
	  // affiche le message "Hello world"
	  int main() {
	      cout <<"Hello world"<<endl;
	      cout << endl;
	      return 0;
	  }
      \end{minted}
      \end{listing}
      
      \begin{listing}[h!]
      \centering{file1.h}
	\begin{minted}[linenos,gobble=2]{cpp}
	   int main();
	\end{minted}
      \end{listing}
      
        \begin{listing}[h!]
        \centering{key1.txt}
	\begin{minted}[linenos,gobble=2]{cpp}
	   int
	   world
	   template
	\end{minted}
      \end{listing}
      
    \end{paragraph}
    
    \begin{paragraph}{Résultat attendu :} 
      Nous lançons le programme avec le contexte suivant :  
       \begin{center}
	\textbf{$tp\_stl\ -e\ -k\ key1.txt\ file1.cpp\ file1.h$}
      \end{center}
      
      Nous devons obtenir le résultat ci-dessous après avoir trié la sortie : 
      \begin{listing}[h!]
          \centering{file1.res}
      \begin{minted}[linenos,
		     gobble=2]{pascal}
	  cout	file1.cpp 3 4	
	  endl	file1.cpp 3 4	
	  main	file1.cpp 2	file1.h 1	
	  return	file1.cpp 5	
      \end{minted}
  
      \end{listing}
    \end{paragraph}

 \end{subsection}
     
   \newpage      
  \begin{subsection}{Test \no 2}
  
    \begin{paragraph}{Descriptif :}
	Le test \no 2 réalise le deuxième exemple donné par le sujet.\\ Les fichiers analysés sont : 
      
      \begin{listing}[h!]
      \centering{file2.cpp}
      \begin{minted}[linenos,
		     gobble=2]{cpp}
	  // affiche le message "Hello world"
	  int main() {
	      cout <<"Hello world"<<endl;
	      cout<<endl;
	      return 0;
	  }
      \end{minted}
      \end{listing}
      
      \begin{listing}[h!]
      \centering{file2.h}
	\begin{minted}[linenos,gobble=2]{cpp}
	   int main();
	\end{minted}
      \end{listing}
      
        \begin{listing}[h!]
        \centering{key2.txt}
	\begin{minted}[linenos,gobble=2]{cpp}
	   int
	   world
	   template
	\end{minted}
      \end{listing}
      
    \end{paragraph}
	   
    \begin{paragraph}{Résultat attendu :}
       Nous lançons le programme avec le contexte suivant après avoir trié la sortie :  
       \begin{center}
	\textbf{$tp\_stl\ -k\ key2.txt\ file2.cpp\ file2.h$}
      \end{center}
      
      Nous devons obtenir le résultat ci-dessous : 
      \begin{listing}[h!]
          \centering{file2.res}
      \begin{minted}[linenos,
		     gobble=2]{pascal}
	  int	file2.cpp 2	file2.h 1	
      \end{minted}
  
      \end{listing}
    \end{paragraph}

  \end{subsection}

   \newpage 
  \begin{subsection}{Test \no 3}
    
    \begin{paragraph}{Descriptif :}
	Le test \no 3 réalise le test sur le fichier main de notre programme.\\ Les fichiers analysés sont : 
      
      \begin{minted}[linenos,
		     gobble=2]{cpp}
	    //============================================================================
	    // Name        : Ref_croisee.cpp
	    // Author      :
	    // Version     :
	    // Copyright   : Your copyright notice
	    // Description : Hello World in C++, Ansi-style
	    //============================================================================

	    #include <iostream>
	    #include <vector>

	    #include "CmdLine/cmdLine.hpp"
	    #include "References/Referenceur.hpp"
	    #include "References/References.hpp"

	    using namespace std;
	    using namespace Reference_croisee;

	    int main( int argc, char** argv )
	    {/*{{{*/

		CmdLine::Arguments args;
		CmdLine::Parser parser( "Permet de referencer des mots clefs a travers des fichiers" );
		parser.addOption( "exclude,e",  "Inverse le fonctionnement du programme" );
		parser.addOption( "keyword,k",  "Specifie la liste des mots clefs a utiliser", true );

		try {
		    parser.parse( argc, argv, args );

		} catch( exception& e ) {
		    cout << "Une erreur c'est produit durant la recuperation de la ligne de commande : "
			 << endl << e.what() << endl;
		}

		//----------------------------------------------------------------------
		//  On charge les fichiers a referencer
		//----------------------------------------------------------------------
		vector<string> ficsReferencer;

		if( args.count( "__args__" ) ) {
		    ficsReferencer = args.get<vector<string> >( "__args__" );

		} else {
		    cerr << "Aucun fichier a referencer !" << endl;
		    return 1;
		}

		//----------------------------------------------------------------------
		//  On charge les mots clefs si ils sont fournis
		//----------------------------------------------------------------------
		string fichierMotClef;

		if( args.count( "keyword" ) ) {
		    fichierMotClef = args.get<string>( "keyword" );
		}

		//----------------------------------------------------------------------
		//  L'etat dans lequel mettre le programme
		//----------------------------------------------------------------------
		bool mode( args.count( "exclude" ) );


		References refs;

		//----------------------------------------------------------------------
		//  On effectue la reference croisee
		//----------------------------------------------------------------------
		try {
		    Referenceur referenceur( fichierMotClef, mode );
		    referenceur.referencer( ficsReferencer, refs );

		} catch( exception& e ) {
		    cerr << "Une erreur est survenue durant la reference croisee : " << endl;
		    cerr << e.what() << endl;
		}


		//----------------------------------------------------------------------
		//  On affiche les resultats
		//----------------------------------------------------------------------
		cout << refs;

		return 0;
	    }/*}}}*/
      \end{minted}
      
    \end{paragraph}
    
    \newpage 
    
    \begin{paragraph}{Résultat attendu :}
       Nous lançons le programme avec le contexte suivant :  
       \begin{center}
	\textbf{$tp\_stl\ file3.cpp$}
      \end{center}
      
      Nous devons obtenir le résultat ci-dessous après avoir trié la sortie: 
      \begin{listing}[h!]
          \centering{file3.res}
      \begin{minted}[linenos,
		     gobble=2]{pascal}
	  bool	file3.cpp 60	
	  catch	file3.cpp 30 72	
	  char	file3.cpp 19	
	  cout	file3.cpp 31 81	
	  else	file3.cpp 43	
	  if	file3.cpp 40 53	
	  int	file3.cpp 19 19	
	  namespace	file3.cpp 16 17	
	  return	file3.cpp 45 83	
	  true	file3.cpp 25	
	  try	file3.cpp 27 68	
	  using	file3.cpp 16 17		
      \end{minted}
  
      \end{listing}
    \end{paragraph}
    
    
  \end{subsection}

  \newpage
  \begin{subsection}{Test \no 4}
    \begin{paragraph}{Descriptif :}
      Le fichier à analyser est le même que dans le test précédent, seul le contexte d'execution change.
    \end{paragraph}
    
\begin{paragraph}{Résultat attendu :}
       Nous lançons le programme avec le contexte suivant :  
       \begin{center}
	\textbf{$tp\_stl\ file4.cpp$}
      \end{center}
      
      Nous devons obtenir le résultat ci-dessous après avoir trié la sortie: 
      \begin{listing}[h!]
          \centering{file4.res}
      \begin{minted}[linenos,
		     gobble=2]{cpp}
	  addOption	file4.cpp 24 25	
	  argc	file4.cpp 19 28	
	  args	file4.cpp 22 28 40 41 53 54 60	
	  Arguments	file4.cpp 22	
	  argv	file4.cpp 19 28	
	  cerr	file4.cpp 44 73 74	
	  CmdLine	file4.cpp 22 23	
	  count	file4.cpp 40 53 60	
	  display	file4.cpp 81	
	  e	file4.cpp 30 32 72 74	
	  endl	file4.cpp 31 32 44 73 74	
	  exception	file4.cpp 30 72	
	  fichierMotClef	file4.cpp 51 54 69	
	  ficsReferencer	file4.cpp 38 41 70	
	  get	file4.cpp 41 54	
	  main	file4.cpp 19	
	  mode	file4.cpp 60 69	
	  parse	file4.cpp 28	
	  Parser	file4.cpp 23	
	  parser	file4.cpp 23 24 25 28	
	  Reference_croisee	file4.cpp 17	
	  referencer	file4.cpp 70	
	  References	file4.cpp 63	
	  Referenceur	file4.cpp 69	
	  referenceur	file4.cpp 69 70	
	  refs	file4.cpp 63 70 81	
	  std	file4.cpp 16	
	  string	file4.cpp 38 41 51 54	
	  vector	file4.cpp 38 41	
	  what	file4.cpp 32 74	
      \end{minted}
  
      \end{listing}
    \end{paragraph}
  \end{subsection}
 
\end{section}




\begin{section}{Sources de l'application}

  \begin{subsection}{Module CmdLine}
  \begin{paragraph}{Parser.hpp}
      \begin{minted}[linenos]{cpp}
// NO_FORMAT ===========================================================================
//
//       Filename:  commandLineParser.hpp
//
//    Description:  Classe permettant d'extraire les informations de la ligne de commande
//        Created:  06/11/2011 00:50:26
//       Compiler:  g++
//
//         Author:  Romain GERARD, romain.gerard@insa-lyon.fr
//
// =====================================================================================


#ifndef CmdLineParser_HPP
#define CmdLineParser_HPP

//------------------------------------------------------------------------Include Systeme
#include    <iostream>
#include    <string>
#include    <map>
#include    <list>
#include    <utility>

namespace CmdLine {

using namespace CmdLine;
class Arguments;


/* =====================================================================================
 *        Class:  Parser
 *  Description:  Permet d'extraire et de decouper les arguments de la ligne de commande
 * =====================================================================================*/
class Parser {

     public:
/*-----------------------------------------------------------------------------
 *  Constructeur
 *-----------------------------------------------------------------------------*/
          Parser( std::string = "" );

/*-----------------------------------------------------------------------------
 *  Methodes Publiques
 *-----------------------------------------------------------------------------*/

          /* ===  FUNCTION  ======================================================================
           *         Name:  AddDescription
           *  Description:  Ajoute une description au programme
           * =====================================================================================*/
          void AddDescription( std::string );

          /* ===  FUNCTION  ======================================================================
           *         Name:  AddOption
           *  Description:  Ajoute une option que doit gerer le programme
           * =====================================================================================*/
          void AddOption( std::string optionName, 
			  std::string description, 
			  bool hasArgument = false );
          
          /* ===  FUNCTION  ======================================================================
           *         Name:  Parse
           *  Description:  Extrait et stocke les informations de la ligne de commande
           * =====================================================================================*/
          void Parse( int argc, char** argv, Arguments& args );

          /* ===  FUNCTION  ======================================================================
           *         Name:  Display
           *  Description:  Affiche la description ainsi que les arguments accepte par le programme
           *                dans le flux specifie
           * =====================================================================================*/
          void Display( std::ostream& flux ) const;


     private:
/*-----------------------------------------------------------------------------
 *  Les etats que peut prendre l'objet
 *-----------------------------------------------------------------------------*/
          enum State { Free, ShortCondition, LongCondition, FreeArgument };
          State _state;

          
/*-----------------------------------------------------------------------------
 *  Les structures de donnees pour stocker les arguments
 *-----------------------------------------------------------------------------*/
          struct optValue;
          typedef std::pair<const std::string, optValue*> lItem;
          typedef std::pair<const char, optValue*> sItem;

          struct optValue {
               std::string description;
               bool hasArgument;
               lItem* lOption;
               sItem* sOption;
          };

          std::map< const std::string, optValue* > _lOptions;
          std::map< const char, optValue* > _sOptions;
          std::list<optValue> _options; // pas de vector car la zone memoire bouge
          std::string _description;
          std::string _freeArgs;

/*-----------------------------------------------------------------------------
 *  Methodes privees
 *-----------------------------------------------------------------------------*/

          /* ===  FUNCTION  ======================================================================
           *         Name:  processInput
           *  Description:  Traite la ligne de commande
           * =====================================================================================*/
          void processInput( std::string::iterator& it, Arguments& args );

          /* ===  FUNCTION  ======================================================================
           *         Name:  changeState
           *  Description:  Determine l'etat de l'objet
           * =====================================================================================*/
          void changeState( std::string::iterator& it );




};

//Surcharge de l'operateur de flux
std::ostream& operator<<( std::ostream& flux,  const Parser& parser );

}/*}}}*/
#endif
      \end{minted}
      \end{paragraph}
   

  \newpage
  \begin{paragraph}{Parser.cpp}
      \begin{minted}[linenos]{cpp}
// =====================================================================================
//
//       Filename:  cmdLineParser.cpp
//
//    Description:  Permet d'extraire les informations de la ligne de commande
//        Created:  06/11/2011 01:43:35
//       Compiler:  g++
//
//         Author:  Romain GERARD, romain.gerard@insa-lyon.fr
//
// =====================================================================================


//------------------------------------------------------------------------Include Systeme
#include    <string>
#include    <map>
#include    <utility>

//------------------------------------------------------------------------Include Personnel
#include    "Parser.hpp"
#include    "Arguments.hpp"

namespace CmdLine {

using namespace CmdLine;
using namespace std;

/*-----------------------------------------------------------------------------
 *  Constructeurs
 *-----------------------------------------------------------------------------*/
Parser::Parser( string description ):
     _state( Free ), _description( description )
{}


/*-----------------------------------------------------------------------------
 *  Methode Publiques
 *-----------------------------------------------------------------------------*/
void Parser::AddDescription( string description )
{/*{{{*/
     _description = description;
}/*}}}*/

void Parser::AddOption( string optionName, string description, bool hasArgument )
{/*{{{*/

     string lOption, sOption;

     size_t pos = optionName.find_first_of( ',' );

     lOption = optionName.substr( 0,  pos );
     sOption = optionName.substr( ( pos == string::npos ) ? optionName.size() : ++pos );

     lItem* lIt = &( *( _lOptions.insert( make_pair( lOption, ( optValue* )0 ) ).first ) );
     sItem* sIt = 0;

     if ( !sOption.empty() ) 
     {	   sIt = &( *( _sOptions.insert( make_pair( sOption.at( 0 ), ( optValue* )0 ) ).first ) );
     }

     optValue value = { description, hasArgument, lIt, sIt };
     _options.push_back( value );

     lIt->second = &( _options.back() );

     if ( sIt ) 
     {     sIt->second = lIt->second;
     }
}/*}}}*/

void Parser::Display( ostream& flux ) const
{/*{{{*/

     string option;
     flux << _description << endl << endl;

     flux << "Liste des arguments : " << endl;

     for ( list<optValue>::const_iterator it = _options.begin(); it != _options.end(); it++ ) 
     {    flux << "\t";

          if( it->sOption ) 
          {     option += "-";
                option += it->sOption->first;
                option += ", ";
          }

          option += "--";
          flux.width( 18 );
          flux << left << option + it->lOption->first;

          flux.width( 5 );

          if ( it->hasArgument ) 
          {     flux << left << "arg";

          }
	  else 
          {     flux << "";
          }

          flux << it->description << endl;

          option.clear();
     }

}/*}}}*/

void Parser::Parse( int argc, char** argv, Arguments& args )
{/*{{{*/

     string cmdLine;

     for ( char** it = argv + 1; it < argv + argc; it++ ) 
     {	  cmdLine += *it;
          cmdLine += " ";
     }

     cmdLine += "  ";

     string::iterator it = cmdLine.begin();

     while ( it != cmdLine.end() ) 
     {    changeState( it );
          processInput( it, args );
     }

     if( !_freeArgs.empty() ) 
     {    args._arguments.insert( make_pair( "__args__", _freeArgs ) );
     }
     //    for( map<string, string>::iterator it = args._arguments.begin();
     //         it != args._arguments.end(); it++ ) {
     //        cout << it->first << " : " << it->second << endl;
     //    }


}/*}}}*/


/*-----------------------------------------------------------------------------
 *  Methodes Privees
 *-----------------------------------------------------------------------------*/
void Parser::changeState( string::iterator& it )
{/*{{{*/

     if ( *it == '-' ) 
     {     _state = ( _state == ShortCondition ) ? LongCondition : ShortCondition;
     }
     else if ( *it == ' ' ) 
     {     _state = Free;
     }
     else if ( _state == ShortCondition )
     {     _state = ShortCondition;
     }
     else
     {     _state = FreeArgument;
     }

}/*}}}*/

void Parser::processInput( string::iterator& it, Arguments& args )
{/*{{{*/


     if ( _state == Free ) 
     {	it++;
     }
     else if ( _state == FreeArgument )

     {	while ( *it != ' ' ) 
        {	_freeArgs += *it;
               	it++;
        }

        _freeArgs += ',';

     } 
     else if ( _state == LongCondition ) 
     {	string key;
        it++;

        while ( *it != '=' && *it != ' ' ) 
        {      key += *it;
               it++;
         }

         if ( !_lOptions.count( key ) ) 
         {      throw UnknownOption( key );
         }

         lItem& item = *( _lOptions.find( key ) );

         if ( ( item.second->hasArgument && *it == ' ' )
              || ( !item.second->hasArgument && *it == '=' ) ) 
         {     throw ArgumentBadFormatted();
         }

         string value;

         if ( !item.second->hasArgument ) 
         {      value = "42";

         }
	 else 
         {      it++;

               while( *it != ' ' )
               {     value += *it;
                    it++;
               }
          }

          if( value.empty() ) 
          {     throw ArgumentBadFormatted();
          }

          args._arguments.insert( make_pair( key, value ) );

     }
     else if ( _state == ShortCondition ) 

     {     if ( *it == '-' ) 
           {    it++;
	   }
     
     	   else 
     	   {	string key;
           	key += *it;

           	if ( !_sOptions.count( *it ) ) 
           	{	throw UnknownOption( key );
           	}

           	sItem& item = *( _sOptions.find( *it ) );
           	key = item.second->lOption->first;
	
           	it++;
	
           	if ( !item.second->hasArgument ) 
           	{	args._arguments.insert( make_pair( key, "42" ) );
           	}
	   	else
	   	{	if ( *it != ' ' ) 
                	{	throw ArgumentBadFormatted();
                	}
	
                	++it;
                	string value;
	
                	while( *it != ' ' ) 
                	{	value += *it;
                	        it++;
                	}
	
                	if( value.empty() ) 
                	{	throw  ArgumentBadFormatted();
                	}
	
                	args._arguments.insert( make_pair( key, value ) );
     	    	}

     	}
     }



}/*}}}*/


/*-----------------------------------------------------------------------------
 *  Surcharge Operateur
 *-----------------------------------------------------------------------------*/
ostream& operator<<( ostream& flux, const Parser& parser )
{/*{{{*/

     parser.Display( flux );

     return flux;
}/*}}}*/

}/*}}}*/
      \end{minted}
      \end{paragraph}

  \newpage
  \begin{paragraph}{Arguments.hpp}
   \begin{minted}[linenos]{cpp}
// NO_FORMAT ============================================================================
//
//       Filename:  cmdLineArgument.hpp
//
//    Description:  Permet de stocker les arguments passe en parametre du programme
//        Created:  06/11/2011 21:35:00
//       Compiler:  g++
//
//         Author:  Romain GERARD, romain.gerard@insa-lyon.fr
//
// =====================================================================================


#ifndef CmdLineArgument_HPP
#define CmdLineArgument_HPP

//------------------------------------------------------------------------Include Systeme
#include    <string>
#include    <map>

//------------------------------------------------------------------------Include Personnel
#include    "Exceptions.hpp"
#include    "StringTo.hpp"

namespace CmdLine {

using namespace CmdLine;

/* =====================================================================================
 *        Class:  Argument
 *  Description:  Permet de gerer les options de la ligne de commande
 * =====================================================================================*/
class Arguments {

     public:
         /* ===  FUNCTION  ======================================================================
          *         Name:  Count
          *  Description:  Permet de savoir si une option est presente
          *                Il faut donner le nom long de l'option
          * =====================================================================================*/
          bool Count( std::string arg ) const {
               return _arguments.count( arg );
          }

          /* ===  FUNCTION  ======================================================================
           *         Name:  Get
           *  Description:  Permet de recuperer une option 
           * =====================================================================================*/
          template<typename T>
          T Get( const char* arg ) {

               if ( !_arguments.count( arg ) )
                    {   throw InvalidKey(); }

               return stringTo<T>( _arguments[arg] );
          }

          /* ===  FUNCTION  ======================================================================
           *         Name:  Get
           *  Description:  Permet de recuperer une option 
           * =====================================================================================*/
          template<typename T>
          T Get( std::string& arg ) {

               if ( !_arguments.count( arg ) )
                    {   throw InvalidKey(); }

               return stringTo<T>( _arguments[arg] );
          }


     private:
          /*-----------------------------------------------------------------------------
           *  Structures de donnees
           *-----------------------------------------------------------------------------*/
          std::map<std::string, std::string> _arguments;

          /*-----------------------------------------------------------------------------
           *  Classe friend
           *-----------------------------------------------------------------------------*/
          friend class Parser;
};

}/*}}}*/

#endif

  \end{minted}
  \end{paragraph}

  \newpage
  \begin{paragraph}{StringTo.hpp}
   \begin{minted}[linenos]{cpp}
// =====================================================================================
// 
//       Filename:  StringTo.hpp
// 
//    Description:  Templates pour convertir les chaines de caracteres
//        Created:  09/11/2011 01:00:03
//       Compiler:  g++
// 
//         Author:  Romain GERARD, romain.gerard@insa-lyon.fr
// 
// =====================================================================================

#ifndef CmdLineStringTo_HPP
#define CmdLineStringTo_HPP

#include	<string>
#include	<vector>
#include	<sstream>
#include	"Exceptions.hpp"


namespace CmdLine {

using namespace CmdLine;

#define STATIC_ASSERT( x , MSG ) typedef char __STATIC_ASSERT__##MSG[( x )?1:-1]

    template<typename T>
        T stringTo( const std::string& arg )
    {/*{{{*/
       
         STATIC_ASSERT( sizeof(T) != sizeof(T),  Impossible_de_convertir_l_argument_vers_ce_type );
         return T();
    }/*}}}*/

/*-----------------------------------------------------------------------------
 *  Types simples
 *-----------------------------------------------------------------------------*/
    template<>
        inline int stringTo<int>( const std::string& arg )
    {/*{{{*/
         std::stringstream s( arg );
         int value = 0;

         s >> value;

         if ( s.fail() )
              { throw BadConvertion(); }

         return value;
    }/*}}}*/

    template<>
        inline float stringTo<float>( const std::string& arg )
    {/*{{{*/
         std::stringstream s( arg );
         float value = 0;

         s >> value;

         if ( s.fail() )
              { throw BadConvertion(); }

         return value;
    }/*}}}*/

    template<>
        inline double stringTo<double>( const std::string& arg )
    {/*{{{*/
         std::stringstream s( arg );
         double value = 0;

         s >> value;

         if ( s.fail() )
              { throw BadConvertion(); }

         return value;
    }/*}}}*/

    template<>
        inline bool stringTo<bool>( const std::string& arg )
    {/*{{{*/
         std::stringstream s( arg );
         int value = 0;

         s >> value;

         if ( s.fail() )
              { throw BadConvertion(); }

         return ( value );
    }/*}}}*/

    template<>
        inline std::string stringTo<std::string>( const std::string& arg )
    {/*{{{*/
         return arg;
    }/*}}}*/


/*-----------------------------------------------------------------------------
 *  Types composes
 *-----------------------------------------------------------------------------*/
    template<>
        inline std::vector<std::string> stringTo<std::vector< std::string> >( const std::string& cpArg )
    {/*{{{*/

         std::string arg = cpArg;
         std::vector<std::string> conteneur;
         const char separator = ',';

         int found = arg.find_first_of( separator );

         while( found != ( ( int ) std::string::npos ) ) {
              if( found > 0 ) {
                   conteneur.push_back( arg.substr( 0, found ) );
              }

              arg = arg.substr( found + 1 );
              found = arg.find_first_of( separator );
         }

         if ( arg.length() > 0 ) {
              conteneur.push_back( arg );
         }

         return conteneur;
    }/*}}}*/
    
    template<>
        inline std::vector<int> stringTo<std::vector<int> >( const std::string& arg )
    {/*{{{*/

         std::vector<int> conteneur;
         const char separator = ',';
         std::stringstream buffer;
         std::string cpArg = arg;

         int found = cpArg.find_first_of( separator );
         int val = 0;

         while( found != ( ( int ) std::string::npos ) ) {

              if( found > 0 ) {
                   buffer << cpArg.substr( 0, found );
                   buffer >> val;
                   
                   if ( buffer.fail() )
                   { throw BadConvertion(); }

                   conteneur.push_back( val );
              }

              cpArg = cpArg.substr( found + 1 );
              found = cpArg.find_first_of( separator );
              buffer.clear();
         }

         if ( arg.length() > 0 ) {
           buffer << cpArg.substr( 0, found );
           buffer >> val;
           if ( buffer.fail() )
           { throw BadConvertion(); }
           conteneur.push_back( val );
         }

         return conteneur;
    }/*}}}*/

    template<>
        inline std::vector<double> stringTo<std::vector<double> >( const std::string& arg )
    {/*{{{*/

         std::vector<double> conteneur;
         const char separator = ',';
         std::stringstream buffer;
         std::string cpArg = arg;

         int found = cpArg.find_first_of( separator );
         double val = 0;

         while( found != ( ( int ) std::string::npos ) ) {

              if( found > 0 ) {
                   buffer << cpArg.substr( 0, found );
                   buffer >> val;
                   
                   if ( buffer.fail() )
                   { throw BadConvertion(); }

                   conteneur.push_back( val );
              }

              cpArg = cpArg.substr( found + 1 );
              found = cpArg.find_first_of( separator );
              buffer.clear();
         }

         if ( arg.length() > 0 ) {
           buffer << cpArg.substr( 0, found );
           buffer >> val;

           if ( buffer.fail() )
           { throw BadConvertion(); }
           
           conteneur.push_back( val );
         }

         return conteneur;
    }/*}}}*/
    
    template<>
        inline std::vector<float> stringTo<std::vector<float> >( const std::string& arg )
    {/*{{{*/

         std::vector<float> conteneur;
         const char separator = ',';
         std::stringstream buffer;
         std::string cpArg = arg;

         int found = cpArg.find_first_of( separator );
         double val = 0;

         while( found != ( ( int ) std::string::npos ) ) {

              if( found > 0 ) {
                   buffer << cpArg.substr( 0, found );
                   buffer >> val;
                   
                   if ( buffer.fail() )
                   { throw BadConvertion(); }

                   conteneur.push_back( val );
              }

              cpArg = cpArg.substr( found + 1 );
              found = cpArg.find_first_of( separator );
              buffer.clear();
         }

         if ( arg.length() > 0 ) {
           buffer << cpArg.substr( 0, found );
           buffer >> val;

           if ( buffer.fail() )
           { throw BadConvertion(); }
           
           conteneur.push_back( val );
         }

         return conteneur;
    }/*}}}*/
}

#endif
  \end{minted}
  \end{paragraph}


  \newpage
  \begin{paragraph}{Exceptions.hpp}
   \begin{minted}[linenos]{cpp}
// =====================================================================================
// 
//       Filename:  StringTo.hpp
// 
//    Description:  Templates pour convertir les chaines de caracteres
//        Created:  09/11/2011 01:00:03
//       Compiler:  g++
// 
//         Author:  Romain GERARD, romain.gerard@insa-lyon.fr
// 
// =====================================================================================

#ifndef CmdLineStringTo_HPP
#define CmdLineStringTo_HPP

//------------------------------------------------------------------------Include Systeme
#include	<string>
#include	<vector>
#include	<sstream>

//------------------------------------------------------------------------Include Personnel
#include	"Exceptions.hpp"


namespace CmdLine {

using namespace CmdLine;

#define STATIC_ASSERT( x , MSG ) typedef char __STATIC_ASSERT__##MSG[( x )?1:-1]

    template<typename T>
        T stringTo( const std::string& arg )
    {/*{{{*/       
         STATIC_ASSERT( sizeof(T) != sizeof(T),  Impossible_de_convertir_l_argument_vers_ce_type );
         return T();
    }/*}}}*/

/*-----------------------------------------------------------------------------
 *  Types simples
 *-----------------------------------------------------------------------------*/
    template<>
        inline int stringTo<int>( const std::string& arg )
    {/*{{{*/
         std::stringstream s( arg );
         int value = 0;

         s >> value;

         if ( s.fail() )
         {	throw BadConvertion();
	 }

         return value;
    }/*}}}*/

    template<>
        inline float stringTo<float>( const std::string& arg )
    {/*{{{*/
         std::stringstream s( arg );
         float value = 0;

         s >> value;

         if ( s.fail() )
         {	throw BadConvertion();
	 }

         return value;
    }/*}}}*/

    template<>
        inline double stringTo<double>( const std::string& arg )
    {/*{{{*/
         std::stringstream s( arg );
         double value = 0;

         s >> value;

         if ( s.fail() )
         {	throw BadConvertion();
	 }

         return value;
    }/*}}}*/

    template<>
        inline bool stringTo<bool>( const std::string& arg )
    {/*{{{*/
         std::stringstream s( arg );
         int value = 0;

         s >> value;

         if ( s.fail() )
         {	throw BadConvertion();
	 }

         return ( value );
    }/*}}}*/

    template<>
        inline std::string stringTo<std::string>( const std::string& arg )
    {/*{{{*/
         return arg;
    }/*}}}*/


/*-----------------------------------------------------------------------------
 *  Types composes
 *-----------------------------------------------------------------------------*/
    template<>
        inline std::vector<std::string> stringTo<std::vector< std::string> >( const std::string& cpArg )
    {/*{{{*/

         std::string arg = cpArg;
         std::vector<std::string> conteneur;
         const char separator = ',';

         int found = arg.find_first_of( separator );

         while ( found != ( ( int ) std::string::npos ) ) 
         {	if( found > 0 ) 
                {	conteneur.push_back( arg.substr( 0, found ) );
                }

              arg = arg.substr( found + 1 );
              found = arg.find_first_of( separator );
         }

         if ( arg.length() > 0 ) 
         {     conteneur.push_back( arg );
         }

         return conteneur;
    }/*}}}*/
    
    template<>
        inline std::vector<int> stringTo<std::vector<int> >( const std::string& arg )
    {/*{{{*/

         std::vector<int> conteneur;
         const char separator = ',';
         std::stringstream buffer;
         std::string cpArg = arg;

         int found = cpArg.find_first_of( separator );
         int val = 0;

         while( found != ( ( int ) std::string::npos ) ) 
	 {	if( found > 0 ) 
                {	buffer << cpArg.substr( 0, found );
                   	buffer >> val;
                   
                   	if ( buffer.fail() )
                   	{	throw BadConvertion();
			}

                   	conteneur.push_back( val );
              	}

              	cpArg = cpArg.substr( found + 1 );
              	found = cpArg.find_first_of( separator );
           	buffer.clear();
         }

         if ( arg.length() > 0 ) 
         {	buffer << cpArg.substr( 0, found );
           	buffer >> val;
           	
		if ( buffer.fail() )
           	{	throw BadConvertion();
		}
           	
		conteneur.push_back( val );
         }

         return conteneur;
    }/*}}}*/

    template<>
        inline std::vector<double> stringTo<std::vector<double> >( const std::string& arg )
    {/*{{{*/

         std::vector<double> conteneur;
         const char separator = ',';
         std::stringstream buffer;
         std::string cpArg = arg;

         int found = cpArg.find_first_of( separator );
         double val = 0;

         while ( found != ( ( int ) std::string::npos ) )
	 {	if ( found > 0 )
                {	buffer << cpArg.substr( 0, found );
                 	buffer >> val;
                   
                   	if ( buffer.fail() )
                   	{	throw BadConvertion(); 
			}

                   conteneur.push_back( val );
    	         }

              	 cpArg = cpArg.substr( found + 1 );
              	 found = cpArg.find_first_of( separator );
              	 buffer.clear();
         }

         if ( arg.length() > 0 ) 
         {	buffer << cpArg.substr( 0, found );
           	buffer >> val;

         	if ( buffer.fail() )
           	{	throw BadConvertion();
		}
           
           	conteneur.push_back( val );
          }

          return conteneur;
    }/*}}}*/
    
    template<>
        inline std::vector<float> stringTo<std::vector<float> >( const std::string& arg )
    {/*{{{*/

         std::vector<float> conteneur;
         const char separator = ',';
         std::stringstream buffer;
         std::string cpArg = arg;

         int found = cpArg.find_first_of( separator );
         double val = 0;

         while ( found != ( ( int ) std::string::npos ) ) 
	 {	if ( found > 0 ) 
                {	buffer << cpArg.substr( 0, found );
                   	buffer >> val;
                   
                   	if ( buffer.fail() )
                   	{	throw BadConvertion();
			}

                   	conteneur.push_back( val );
              	}

              	cpArg = cpArg.substr( found + 1 );
              	found = cpArg.find_first_of( separator );
              	buffer.clear();
         }

         if ( arg.length() > 0 )
         {	buffer << cpArg.substr( 0, found );
           	buffer >> val;

           	if ( buffer.fail() )
           	{	throw BadConvertion();
		}
           
           	conteneur.push_back( val );
         }

         return conteneur;
    }/*}}}*/
}

#endif
  \end{minted}
  \end{paragraph}

  \end{subsection}

  \newpage
  \begin{subsection}{Module References}
   
  \begin{paragraph}{References.hpp}
   \begin{minted}[linenos]{cpp}
// =====================================================================================
//
//       Filename:  References.hpp
//
//    Description:  Permet de stocker les resultats des references croisees
//        Created:  18/11/2011 22:29:34
//       Compiler:  g++
//
//         Author:  Romain GERARD, romain.gerard@insa-lyon.fr
//
// =====================================================================================


#ifndef References_HPP
#define References_HPP

//------------------------------------------------------------------------Include Systeme
#include    <map>
#include    <list>
#include    <tr1/unordered_map>

namespace Reference_croisee {

using namespace Reference_croisee;

//------------------------------------------------------------------------
// Role de la classe References
// Description : Permet de stocker les occurences des references croisees
//
//------------------------------------------------------------------------
class References {

    public:
//----------------------------------------------------------------------
//  METHODES PUBLIQUES
//----------------------------------------------------------------------

        // ===  FUNCTION  ======================================================================
        //         Name:  Add
        //  Description:  Permet d'ajouter un motClef passe en parametre rencontre dans le fichier
	//		  nomFichier et a la ligne au conteneur
        // =====================================================================================
        void Add( const std::string& motClef, const std::string& nomFichier, int ligne );

        // ===  FUNCTION  ======================================================================
        //         Name:  Display
        //  Description:  Permet d'afficher toutes les references dans le flux fournit en entree
        // =====================================================================================
        void Display( std::ostream& flux ) const;


    protected:
//----------------------------------------------------------------------
//  ATTRIBUTS MEMBRES
//----------------------------------------------------------------------

        // Structure interne du conteneur
        std::tr1::unordered_map<std::string, std::map<std::string, std::list<int> > > _references;

};

//----------------------------------------------------------------------
//  SURCHARGES OPERATEURS
//----------------------------------------------------------------------

std::ostream& operator<<( std::ostream& flux,  const References& ref );

}/*}}}*/

#endif

  \end{minted}
  \end{paragraph}



\newpage
  \begin{paragraph}{References.cpp}
   \begin{minted}[linenos]{cpp}
// =====================================================================================
//
//       Filename:  References.cpp
//
//    Description:
//        Created:  18/11/2011 22:50:44
//       Compiler:  g++
//
//         Author:  Romain GERARD, romain.gerard@insa-lyon.fr
//
// =====================================================================================


//---------------------------------------------------------------------- include systeme
#include <iostream>

//---------------------------------------------------------------------- include personnel
#include "References.hpp"

namespace Reference_croisee {

using namespace std;
using namespace Reference_croisee;


//----------------------------------------------------------------------
//  METHODES PUBLIQUES
//----------------------------------------------------------------------
void References::Add( const string& motClef, const string& nomFichier, const int ligne )
{/*{{{*/


    if ( !_references.count( motClef ) )
    {	_references.insert( make_pair( motClef, map<string, list<int> >() ) );
        _references[motClef].insert( make_pair( nomFichier, list<int>( 1, ligne ) ) );
    }
    else if ( !_references[motClef].count( nomFichier ) )
    {	_references[motClef].insert( make_pair( nomFichier, list<int>( 1, ligne ) ) );
    }
    else
    {	_references[motClef][nomFichier].push_back( ligne );
    }


}/*}}}*/

void References::Display( ostream& flux ) const
{/*{{{*/

    tr1::unordered_map<string, map<string, list<int> > >::const_iterator itClef;
    map<string, list<int> >::const_iterator itFic;
    list<int>::const_iterator itLigne;


    for ( itClef = _references.begin(); itClef != _references.end(); itClef++ )
    {	//flux.width(15);
        //flux << left;
        flux << itClef->first << "\t";

        for ( itFic = itClef->second.begin(); itFic != itClef->second.end(); itFic++ )
	{	flux << itFic->first;

            for ( itLigne = itFic->second.begin(); itLigne != itFic->second.end(); itLigne++ )
	    {	flux << ' ' << *itLigne;
            }

            flux << '\t';
        }

        flux << endl;
    }

}/*}}}*/



//----------------------------------------------------------------------
//  SURCHARGES OPERATEURS
//----------------------------------------------------------------------
ostream& operator<<( ostream& flux,  const References& ref ) {

    ref.Display( flux );

    return flux;

}

}
  \end{minted}

  \end{paragraph}





\newpage
  \begin{paragraph}{Referenceur.hpp}
   \begin{minted}[linenos]{cpp}
// =====================================================================================
//
//       Filename:  References.hpp
//
//    Description:  Interface de la classe References
//                  Permet de visualiser la repartition de mots clefs dans une
//                  collection de fichiers
//        Created:  15/11/2011 23:30:30
//       Compiler:  g++
//
//         Author:  Romain GERARD, romain.gerard@insa-lyon.fr
//
// =====================================================================================


#ifndef Referenceur_HPP
#define Referenceur_HPP

//------------------------------------------------------------------------Include Systeme
#include    <string>
#include    <vector>
#include    <tr1/unordered_set>

//------------------------------------------------------------------------Include Personnels
#include    "FichierLu.hpp"
#include    "References.hpp"

namespace Reference_croisee {

using namespace Reference_croisee;

//------------------------------------------------------------------------
// Role de la classe References
// Description : Permet de visualiser la repartition de mots clefs dans une
//               collection de fichiers
//------------------------------------------------------------------------
class Referenceur {

    public:
//----------------------------------------------------------------------
//  CONSTRUCTEURS
//----------------------------------------------------------------------
        Referenceur( const std::string fichierMotClef = std::string(),
                     const bool modeInverse = false );

//----------------------------------------------------------------------
//  METHODES PUBLIQUES
//----------------------------------------------------------------------

        // ===  FUNCTION  ======================================================================
        //         Name:  ChargerMotsClefs
        //  Description:  Permet de charger des mots clefs a partir d'un fichier "nomFichier" 
        // =====================================================================================
        void ChargerMotsClefs( const std::string& nomFichier );
        
        // ===  FUNCTION  ======================================================================
        //         Name:  ChargerMotsClefsCpp
        //  Description:  Permet de definir les mots clefs C++ standard comme des mots clefs
        // =====================================================================================
        void ChargerMotsClefsCpp();
        
        // ===  FUNCTION  ======================================================================
        //         Name:  SetModeInverse
        //  Description:  Permet de passer le parseur en mode inverse
        // =====================================================================================
        inline void SetModeInverse( const bool mode );

        // ===  FUNCTION  ======================================================================
        //         Name:  Referencer
        //  Description:  Permet de chercher les mots clefs une collection de fichier
        //                Les resultats sont stockes dans refs
        // =====================================================================================
        void Referencer( const std::vector<std::string>& fic, References& refs );

    protected:
        enum Mode { Normal, Inverse };          // Les differents mode du parseur
        enum Etat { Separateur, Commentaire, MotClef, Preprocesseur, Literal }; // Les etats interne de l'automate



//----------------------------------------------------------------------
//  ATTRIBUTS MEMBRES
//----------------------------------------------------------------------
        Mode _mode;
        Etat _etat;

        std::tr1::unordered_set<std::string> _motsClefs; // le conteneur des mots clefs



//----------------------------------------------------------------------
//  METHODES PROTEGES
//----------------------------------------------------------------------

        // ===  FUNCTION  ======================================================================
        //         Name:  estInserable
        //  Description:  retourne vrai si l'identificateur passe en parametre est un mot clef
        // =====================================================================================
        inline bool estInserable( const std::string& mot ) const;

        // ===  FUNCTION  ======================================================================
        //         Name:  isSeparateur
        //  Description:  Retourne vrai si le caractere c est un separateur
        // =====================================================================================
        bool isSeparateur( const char c ) const;

        // ===  FUNCTION  ======================================================================
        //         Name:  changerEtat
        //  Description:  Definit le nouvel etat de l'automate
        // =====================================================================================
        void changerEtat( FichierLu& fic );

        // ===  FUNCTION  ======================================================================
        //         Name:  lireFlux
        //  Description:  Avance dans le flux de donnees en fonction de l'etat de l'automate
        // =====================================================================================
        void lireFlux( FichierLu& fic, References& refs );



//----------------------------------------------------------------------
//  METHODES ETATS
//----------------------------------------------------------------------

        // ===  FUNCTION  ======================================================================
        //         Name:  lirePreprocesseur
        //  Description:  Traite les instructions preprocesseurs
        // =====================================================================================
        void lirePreprocesseur( FichierLu& fic, References& refs );

        // ===  FUNCTION  ======================================================================
        //         Name:  lireSeparateur
        //  Description:  Traite les separateurs
        // =====================================================================================
        void lireSeparateur( FichierLu& fic, References& refs );

        // ===  FUNCTION  ======================================================================
        //         Name:  lireIdentificateur
        //  Description:  Traite les identificateurs
        // =====================================================================================
        void lireIdentificateur( FichierLu& fic, References& refs );

        // ===  FUNCTION  ======================================================================
        //         Name:  lireCommentaire
        //  Description:  Traite les commentaires
        // =====================================================================================
        void lireCommentaire( FichierLu& fic, References& refs );

        // ===  FUNCTION  ======================================================================
        //         Name:  lireLiteral
        //  Description:  Traire les chaines de caracteres
        // =====================================================================================
        void lireLiteral( FichierLu& fic, References& refs );


};

}/*}}}*/
#endif

  \end{minted}
  \end{paragraph}



\newpage
  \begin{paragraph}{Referenceur.cpp}
   \begin{minted}[linenos]{cpp}
// =====================================================================================
//
//       Filename:  Referenceur.cpp
//
//    Description:  Implementation de la classe Referenceur
//                  Permet de visualiser l'aparition de mots clefs dans une collection
//                  de fichiers
//        Created:  15/11/2011 23:37:07
//       Compiler:  g++
//
//         Author:  Romain GERARD, romain.gerard@insa-lyon.fr
//
// =====================================================================================

using namespace std;

//------------------------------------------------------------------------Include Systeme
#include    <iostream>
#include    <limits>

//------------------------------------------------------------------------Include Personnel
#include    "Referenceur.hpp"

namespace Reference_croisee {

using namespace Reference_croisee;



//----------------------------------------------------------------------
//  CONSTRUCTEURS
//----------------------------------------------------------------------
Referenceur::Referenceur ( const string fichierMotClef, const bool modeInverse ):
    _mode( modeInverse ? Inverse : Normal ), _etat( Separateur )
{/*{{{*/

#ifdef MAP
    cout << "Appel au constructeur de <Referenceur>" << endl;
#endif

    ChargerMotsClefsCpp();

    if ( !fichierMotClef.empty() ) 
    {    ChargerMotsClefs( fichierMotClef );
    }

}/*}}}*/

//----------------------------------------------------------------------
//  METHODES PUBLIQUES
//----------------------------------------------------------------------
void Referenceur::ChargerMotsClefs( const string& nomFichier )
{/*{{{*/


    ifstream fichierMotClef;

    //----------------------------------------------------------------------
    //  Si le fichier ne peut etre ouvert ou si la lecture echoue
    //  une exception sera lance
    //----------------------------------------------------------------------
    fichierMotClef.exceptions( ifstream::failbit );
    fichierMotClef.open( nomFichier.c_str(), ios::in );
    fichierMotClef.exceptions( ifstream::badbit );

    //----------------------------------------------------------------------
    //  On extrait la liste de mot clef
    //----------------------------------------------------------------------
    _motsClefs.clear();
    string motRecupere;

    while ( fichierMotClef >> motRecupere ) 
    {    _motsClefs.insert( motRecupere );
        fichierMotClef.ignore( numeric_limits<int>::max(), '\n' );
        // ignore le nombre de caractere "valeur max d'un entier" jusqu'a rencontrer \n
    }

    fichierMotClef.close( );

}/*}}}*/

void Referenceur::ChargerMotsClefsCpp()
{/*{{{*/

    _motsClefs.clear();
    _motsClefs.insert( "asm" );
    _motsClefs.insert( "auto" );
    _motsClefs.insert( "break" );
    _motsClefs.insert( "bool" );
    _motsClefs.insert( "case" );
    _motsClefs.insert( "catch" );
    _motsClefs.insert( "cout" );
    _motsClefs.insert( "char" );
    _motsClefs.insert( "class" );
    _motsClefs.insert( "const" );
    _motsClefs.insert( "const_cast" );
    _motsClefs.insert( "continue" );
    _motsClefs.insert( "default" );
    _motsClefs.insert( "delete" );
    _motsClefs.insert( "do" );
    _motsClefs.insert( "double" );
    _motsClefs.insert( "dynamic_cast" );
    _motsClefs.insert( "else" );
    _motsClefs.insert( "enum" );
    _motsClefs.insert( "extern" );
    _motsClefs.insert( "export" );
    _motsClefs.insert( "explicit" );
    _motsClefs.insert( "false" );
    _motsClefs.insert( "float" );
    _motsClefs.insert( "for" );
    _motsClefs.insert( "friend" );
    _motsClefs.insert( "goto" );
    _motsClefs.insert( "if" );
    _motsClefs.insert( "inline" );
    _motsClefs.insert( "int" );
    _motsClefs.insert( "long" );
    _motsClefs.insert( "mutable" );
    _motsClefs.insert( "namespace" );
    _motsClefs.insert( "new" );
    _motsClefs.insert( "operator" );
    _motsClefs.insert( "private" );
    _motsClefs.insert( "protected" );
    _motsClefs.insert( "public" );
    _motsClefs.insert( "register" );
    _motsClefs.insert( "reinterpret_cast" );
    _motsClefs.insert( "return" );
    _motsClefs.insert( "short" );
    _motsClefs.insert( "signed" );
    _motsClefs.insert( "sizeof" );
    _motsClefs.insert( "static" );
    _motsClefs.insert( "static_cast" );
    _motsClefs.insert( "struct" );
    _motsClefs.insert( "switch" );
    _motsClefs.insert( "template" );
    _motsClefs.insert( "this" );
    _motsClefs.insert( "throw" );
    _motsClefs.insert( "try" );
    _motsClefs.insert( "true" );
    _motsClefs.insert( "typedef" );
    _motsClefs.insert( "typeid" );
    _motsClefs.insert( "typename" );
    _motsClefs.insert( "unsigned" );
    _motsClefs.insert( "union" );
    _motsClefs.insert( "using" );
    _motsClefs.insert( "virtual" );
    _motsClefs.insert( "void" );
    _motsClefs.insert( "volatile" );
    _motsClefs.insert( "while" );
    _motsClefs.insert( "wchar_t" );
}/*}}}*/

void Referenceur::Referencer( const vector<string>& fichiers, References& refs )
{/*{{{*/

    vector<string>::const_iterator it;
    FichierLu fichier;

    for ( it = fichiers.begin(); it != fichiers.end(); it++ ) 
    {    fichier.open( it->c_str() );

        while( !fichier.eof() ) 
        {    changerEtat( fichier );
            lireFlux( fichier, refs );
        }

        fichier.close();
    }

}/*}}}*/

inline void Referenceur::SetModeInverse( const bool mode )
{/*{{{*/
    _mode = ( mode ) ? Inverse : Normal;
}/*}}}*/



//----------------------------------------------------------------------
//  METHODES PROTEGES
//----------------------------------------------------------------------
inline bool Referenceur::estInserable( const string& mot ) const
{/*{{{*/

    const char c = mot.at( 0 );

    if ( c >= '0' && c <= '9' ) 
    {    return false;
    }

    return  ( _mode == Normal ) ? _motsClefs.count( mot ) :
            !_motsClefs.count( mot );


}/*}}}*/

inline bool Referenceur::isSeparateur( const char c ) const
{/*{{{*/

    // Je me suis base sur la table ASCII
    return ( c >= -1 && c <= '/' ) ||
           ( c >= ':' && c <= '@' ) ||
           ( c >= '[' && c <= '^' ) ||
           ( c >= '{' && c <= '~' ) ||
           ( c == '`' );

}/*}}}*/

void Referenceur::changerEtat( FichierLu& fic )
{/*{{{*/

    const char c = fic.peek();

    if ( c == '#' ) 
    {    _etat = Preprocesseur;
    }
    else if (  c == '/' ) 
    {    _etat = Commentaire;
    }
    else if (  c == '"' || c == '\'' ) 
    {    _etat = Literal;
    }
    else if ( isSeparateur( c ) )
    {	_etat = Separateur;
    }
    else 
    {    _etat = MotClef;
    }
}/*}}}*/

void Referenceur::lireFlux( FichierLu& fic, References& refs )
{/*{{{*/

    switch( _etat )
    {

        case Preprocesseur:
            lirePreprocesseur( fic, refs );
            break;

        case Separateur:
            lireSeparateur( fic, refs );
            break;

        case Commentaire:
            lireCommentaire( fic, refs );
            break;

        case MotClef:
            lireIdentificateur( fic, refs );
            break;

        case Literal:
            lireLiteral( fic, refs );
            break;
    }

}/*}}}*/




//----------------------------------------------------------------------
//  METHODES ETATS
//----------------------------------------------------------------------
void Referenceur::lirePreprocesseur( FichierLu& fic, References& refs )
{/*{{{*/

    char last = fic.get();

    while ( fic.peek() != '\n' || last == '\\' )
    {    last = fic.get();
    }

    fic.get();

}/*}}}*/

void Referenceur::lireCommentaire( FichierLu& fic, References& refs )
{/*{{{*/

    fic.get();

    if( fic.peek() == '/' ) 
    {    while ( !fic.eof() && fic.get() != '\n' )
	 {}// bloc vide

    }
    else if( fic.peek() == '*' ) 
    {    while ( !fic.eof() && ( fic.get() != '*' || fic.peek() != '/' ) )
	 {} // bloc vide

        fic.get();
    }

}/*}}}*/

void Referenceur::lireIdentificateur( FichierLu& fic, References& refs )
{/*{{{*/

    string mot;
    mot.append( 1, fic.get() );

    while( !fic.eof() && !isSeparateur( fic.peek() ) ) 
    {    mot.append( 1, fic.get() );
    }

    if( estInserable( mot ) ) 
    {    refs.Add( mot, fic.GetNomFichier(), fic.GetNbLignesLues() );
    }

}/*}}}*/

void Referenceur::lireSeparateur( FichierLu& fic, References& refs )
{/*{{{*/
    fic.get();
}/*}}}*/

void Referenceur::lireLiteral( FichierLu& fic, References& refs )
{/*{{{*/

    char last = fic.get();

    if( last == '"' ) 
    {    while( fic.peek() != '"' || last == '\\' ) 
        {    last = fic.get();
        }

        fic.get();

    }
    else if( last == '\'' ) 
    {    while( fic.peek() != '\'' || last == '\\' ) 
         {    last = fic.get();
         }

        fic.get();
    }

}/*}}}*/

}/*}}}*/
  \end{minted}
  \end{paragraph}





\newpage
  \begin{paragraph}{FichierLu.hpp}
   \begin{minted}[linenos]{cpp}/*************************************************************************
                           FichierLu  -  description
                             -------------------
    debut                : 18 nov. 2011
    copyright            : (C) 2011 par csaysset
*************************************************************************/

#ifndef FICHIERLU_HPP_
#define FICHIERLU_HPP_

//------------------------------------------------------------------------Include Systeme
#include    <fstream>
#include    <string>


namespace Reference_croisee {

using namespace Reference_croisee;

/* =====================================================================================
 *        Class:  FichierLu
 *  Description:  Permet de lire un fichier en concervant son nom et le nombre de ligne
 *                deja parcourue
 * =====================================================================================*/
class FichierLu : private std::ifstream {

    public:
        // ===  FUNCTION  ======================================================================
        //         Name:  FichierLu
        //  Description:  Constructeur de la classe, prend en argument un chemin vers un fichier
        // =====================================================================================
        FichierLu ( const std::string& nomFichier = "" );





//----------------------------------------------------------------------
//  METHODES MASQUEES
//----------------------------------------------------------------------

        // ===  FUNCTION  ======================================================================
        //         Name:  Close
        //  Description:  Ferme le fichier
        // =====================================================================================
        void close();

        // ===  FUNCTION  ======================================================================
        //         Name:  open
        //  Description:  Ouvre un fichier
        // =====================================================================================
        void open( const char* filename,  ios_base::openmode mode = ios_base::in );


        // ===  FUNCTION  ======================================================================
        //         Name:  get
        //  Description:  Permet de recuperer un caractere 
        // =====================================================================================
        int get();

        int peek() { return std::ifstream::peek(); }
        bool eof() { return std::ifstream::eof(); }




//----------------------------------------------------------------------
//  METHODES PUBLIQUES
//----------------------------------------------------------------------

        // ===  FUNCTION  ======================================================================
        //         Name:  GetNbLignesLues
        //  Description:  Retourne le nombre de lignes deja lues dans le fichier
        // =====================================================================================
        int GetNbLignesLues() const;

        // ===  FUNCTION  ======================================================================
        //         Name:  GetNomFichier
        //  Description:  Retourne le nom du fichier passe lors de la construction de l'objet 
        // =====================================================================================
        std::string GetNomFichier() const;


    protected:
        int _nbLignesLues;                      // contient le nombre de lignes lues
        std::string _nomFichier;                // contient le nom du fichier
};

}/*}}}*/

#endif
\end{minted}
  \end{paragraph}



\newpage
  \begin{paragraph}{FichierLu.cpp}
   \begin{minted}[linenos]{cpp}
//*************************************************************************
                           FichierLu  -  description
                             -------------------
    debut                : 18 nov. 2011
    copyright            : (C) 2011 par csaysset
*************************************************************************/

//------------------------------------------------------------------------Include Systeme
#include <fstream>

//------------------------------------------------------------------------Include Personnel
#include "FichierLu.hpp"


namespace Reference_croisee {

using namespace std;
using namespace Reference_croisee;




//----------------------------------------------------------------------
//  CONSTRUCTEURS
//----------------------------------------------------------------------
FichierLu::FichierLu ( const string& nomFichier ):
    _nbLignesLues( 1 )
{/*{{{*/

    if ( !nomFichier.empty() ) 
    {    open( nomFichier.c_str() );
    }
}/*}}}*/




//----------------------------------------------------------------------
//  METHODES MASQUEES
//----------------------------------------------------------------------
void FichierLu::close()
{/*{{{*/

    ifstream::close();
    ifstream::clear();

    _nbLignesLues = 1;
    _nomFichier.clear();
}/*}}}*/

void FichierLu::open( const char* filename,  ios_base::openmode mode )
{/*{{{*/

    exceptions( ifstream::failbit );
    ifstream::open( filename, mode );
    exceptions( ifstream::badbit );
    _nomFichier = filename;
}/*}}}*/

int FichierLu::get()
{/*{{{*/

    int charactere = ifstream::get();

    if ( charactere == '\n' ) 
    {	_nbLignesLues++;
    }

    return charactere;
}/*}}}*/



//----------------------------------------------------------------------
// METHODES PUBLIQUES
//----------------------------------------------------------------------
int FichierLu::GetNbLignesLues() const
{/*{{{*/
    return _nbLignesLues;
}/*}}}*/

string FichierLu::GetNomFichier() const
{/*{{{*/
    return _nomFichier;
}/*}}}*/

}/*}}}*/
  \end{minted}
  \end{paragraph}


\end{subsection}


  \newpage
  \begin{subsection}{Module Main}
   
  \begin{paragraph}{Ref\_croisee.cpp}
   \begin{minted}[linenos]{cpp}//============================================================================
// Name        : Ref_croisee.cpp
// Author      :
// Version     :
// Copyright   : Your copyright notice
// Description : Hello World in C++, Ansi-style
//============================================================================

#include <iostream>
#include <vector>

#include "CmdLine/cmdLine.hpp"
#include "References/Referenceur.hpp"
#include "References/References.hpp"

using namespace std;
using namespace Reference_croisee;

int main( int argc, char** argv )
{/*{{{*/

    CmdLine::Arguments args;
    {
        CmdLine::Parser parser( "Permet de referencer des mots clefs a travers des fichiers\n\t" + string(argv[0]) + " [options] fichier+" );
        parser.AddOption( "help,h",  "Affiche ce message" );
        parser.AddOption( "exclude,e",  "Inverse le fonctionnement du programme" );
        parser.AddOption( "keyword,k",  "Specifie la liste des mots clefs a utiliser", true );

        try {
            parser.Parse( argc, argv, args );

        } catch( exception& e ) {
            cout << e.what() << endl;
            cout << parser << endl;
            return 1;
        }

        if( !args.Count( "__args__" ) || args.Count( "help" ) ) {

            cout << parser << endl;
            return 1;
        }
    }

    //----------------------------------------------------------------------
    //  On charge les fichiers a referencer
    //----------------------------------------------------------------------
    vector<string> ficsReferencer;
    ficsReferencer = args.Get<vector<string> >( "__args__" );

    //----------------------------------------------------------------------
    //  On charge les mots clefs si ils sont fournis
    //----------------------------------------------------------------------
    string fichierMotClef;

    if( args.Count( "keyword" ) ) {
        fichierMotClef = args.Get<string>( "keyword" );
    }

    //----------------------------------------------------------------------
    //  L'etat dans lequel mettre le programme
    //----------------------------------------------------------------------
    bool mode( args.Count( "exclude" ) );


    //----------------------------------------------------------------------
    //  On effectue la reference croisee
    //----------------------------------------------------------------------
    References refs;
    try {
        Referenceur referenceur( fichierMotClef, mode );
        referenceur.Referencer( ficsReferencer, refs );

    } catch( exception& e ) {
        cerr << "Une erreur est survenue durant la referance croisee : " << endl;
        cerr << e.what() << endl;
    }


    //----------------------------------------------------------------------
    //  On affiche les resultats
    //----------------------------------------------------------------------
    cout << refs; 

    return 0;
}/*}}}*/
  \end{minted}
  \end{paragraph}

  \end{subsection}

\end{section}



\end{document}
